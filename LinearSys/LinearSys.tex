\documentclass[10pt]{ctexart}
\usepackage{NotesTeX}

%\usepackage{showframe}

\title{\begin{center}{\Huge \textit{信号与系统}}\\{{\itshape LinearSystem}}\end{center}}
\author{Seastian Ludwig\footnote{\href{https://geodesick.com/}{\textit{My Personal Website}}}}


\affiliation{
   University of Northeast Electronic Power University\\
Electronic Engineer school\\
}

\emailAdd{2019301011016@neepu.edu.cn}
\begin{document}
\maketitle
\flushbottom
\newpage
\pagestyle{fancynotes}

\part{1-1}
\section{确定信号和随机信号}
\begin{definition}

   确定信号:$$	f(t)$$ 
   随机信号
\end{definition}
\section{连续信号和离散信号}
\section{周期信号和非周期信号}
\begin{definition}
   连续周期信号:
   $$	f(t)=f(t+mT),m=0,\pm 1,\pm 2,\dots$$ 
   离散周期信号:
   $$	f(k)=f(k+mN),m=0,\pm 1,\pm 2,\dots$$ 
\end{definition}
\begin{margintable}
   满足上述式子的最小T称为周期

   $$	f(k)=\sin(\beta k)=\sin(\beta k+2m\pi)=\\ \sin[\beta (k+m\frac{2\pi}{\beta})]$$ 
\end{margintable}
\begin{lemma}
   \begin{enumerate}
      \item $ \frac{2\pi}{\beta} $ 为整数,正弦序列才具有周期$ N=\frac{2\pi}{\beta} $ 
      \item  $ \frac{2\pi}{\beta} $ 为有理数,周期为$ N=M\frac{2\pi}{\beta} $ 
   \end{enumerate}
\end{lemma}
\begin{example}
   \begin{enumerate}
      \item $ f_{1}(k)=sin(\frac{\pi}{7}k+\frac{\pi}{6})  $ 
      \item $ f_{2}(k)  =\cos(\frac{5\pi}{6}k+\frac{\pi}{12})$ 
      \item $ f_{3} (k)= $ 
   \end{enumerate}
\end{example}
\section{实信号和复信号}
\section{能量信号和功率信号}
\section{信号的基本运算}
\subsection{加法乘法}
\end{document}